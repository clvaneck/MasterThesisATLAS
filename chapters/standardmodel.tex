\chapter{The Standard Model}
\label{chap:sm}
The Standard Model, illustrated in figure \ref{fig:SM}, is a quantum field theory describing the known elementary particles and the strong, weak, and electromagnetic interactions. Its structure is determined by the gauge symmetry
\begin{equation}
SU(3)_{C}\times SU(2)_{L}\times U(1)_{Y},
\end{equation}
which contains the interaction terms and fixes the form  of the Lagrangian. The strong interaction is governed by the non-Abelian colour group $SU(3)_{C}$, while the electroweak theory unifies the weak and electromagnetic forces through the groups $SU(2)_{L}$ and $U(1)_{Y}$.
\begin{figure}[h]
    \centering
    \includegraphics[scale=0.15]{figures/Standard.png}
    \caption{The Standard Model of particle physics. Showing the quarks (purple), leptons (green), gauge bosons (red) and the Higgs boson (yellow).}
    \label{fig:SM}
\end{figure}

Fermions, which are spin-$\tfrac{1}{2}$ particles, make up matter. They are arranged into three generations of quark and leptons. Quarks carry colour charge and transform as triples under $SU(3)_{C}$, leading to confinement and the formation of hadrons. Leptons are colourless and interact only through the electroweak sector. The weak interaction is intrinsically chiral, left handed fermions transform as doublets under $SU(2)_{L}$, while right handed fermions are singlets. Fermion masses arise from Yukawa interactions with the Higgs field after electroweak symmetry breaking.

Gauge bosons mediate the fundamental forces. Quantum chromodynamics (QCD) contains eight gluons associated with the symmetry of $SU(3)_{C}$. Gluons also carry colour charge and undergo self interaction characteristic of non-Abelian gauge symmetries, leading to phenomena such as Asymptotic freedom. In the electroweak sector, the gauge fields associated with $SU(2)_{L}$ and $U(1)_{Y}$ mix after symmetry breaking to produce the physical gauge bosons, the massless photon and the massive weak bosons $W^{\pm}$ and $Z$.

The Higgs field is a scalar $SU(2)_{L}$ doublet whose nonzero vacuum expectation value breaks the electroweak symmetry spontaneously. This mechanism generates masses for the weak gauge bosons and the fermions while preserving gauge invariance. The physical excitation of the field is the Higgs boson, observed at the LHC in 2012, which completed the particle spectrum predicted by the SM.

Although experimentally successful, the SM is incomplete. It does not incorporate a quantum theory of gravity, does not account for dark matter or the observed matter–antimatter asymmetry, and must be extended to accommodate nonzero neutrino masses inferred from neutrino oscillations. Nevertheless, the SM remains the most accurate and extensively tested description of particle interactions at currently accessible energies.

\section{Vector Boson Scattering}
Vector boson scattering (VBS) is an electroweak process in which two incoming quarks emit weak gauge bosons that subsequently interact, as seen in fig. \ref{fig:vbs}. The characteristic topology consists of two forward jets produced by the scattered quarks and two bosons that decay either leptonically or hadronically. The fully hadronic channel, in which both bosons decay into quarks, has the largest branching fraction but is experimentally challenging due to the overwhelming background, such as from QCD multijet production.

\begin{figure}[h]
    \centering
    \includegraphics[scale=0.4]{figures/vbs.png}
    \caption{Feynman diagram of the vector boson scattering process. Needs to change to better picture}
    \label{fig:vbs}
\end{figure}

An important feature of VBS is its sensitivity to the electroweak symmetry breaking mechanism. in the SM the longitudinal components of the gauge bosons arise from the Higgs field, and the scattering amplitude for longitudinally polarised bosons exhibits delicate cancellations between diagrams involving the Higgs exchange and those involving triple and quartic gauge interactions. These cancellations preserve perturbative unitarity at high energies. Any modification of the electroweak sector may disrupt these cancellations, leading to an enhanced scattering rate or distortions in kinematic distributions, particularly at high invariant masses.



\section{Beyond the Standard Model}
\section{Effective Field Theory}
\subsection{Standard Model Effective Field Theory}
\subsection{Anomalous Quartic Gauge Couplings}