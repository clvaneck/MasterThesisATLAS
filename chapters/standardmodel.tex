\chapter{The Standard Model}
\label{chap:sm}
The Standard Model of particle physics is a quantum field theory describing the known elementary particles and the strong, weak, and electromagnetic interactions \cite{Thomson_ModernParticlePhysics_2013, Griffiths_IntroParticles_2008}. Its a Yang-Mills gauge theory \cite{YangMills_Symmetry_1954} determined by the gauge symmetry
\begin{equation}
SU(3)_{C}\times SU(2)_{L}\times U(1)_{Y},
\end{equation}
which contains the interaction terms and fixes the form  of the Lagrangian. The strong interaction is governed by the non-Abelian colour group $SU(3)_{C}$, while the electroweak theory unifies the weak and electromagnetic forces through the groups $SU(2)_{L}$ and $U(1)_{Y}$ \cite{Feynman_QED_1950, Glashow_EW_1961, Salam_EW_1968}. The particle content of the SM is illustrated in fig~\ref{fig:SM}.
\begin{figure}[h]
    \centering
    \includegraphics[scale=0.085]{figures/Standard.png}
    \caption{The particle content of the Standard Model of particle physics. Showing the quarks (purple), leptons (green), gauge bosons (red) and the Higgs boson (yellow) \cite{Neutelings_SM_2016}.}
    \label{fig:SM}
\end{figure}

Fermions, which are spin-$\tfrac{1}{2}$ particles, make up matter. They are arranged into three generations of quark and leptons. Quarks carry colour charge and transform as triplets under $SU(3)_{C}$, leading to confinement and the formation of hadrons. Leptons are colourless and interact only through the electroweak sector. The weak interaction is intrinsically chiral, left handed fermions transform as doublets under $SU(2)_{L}$, while right handed fermions are singlets. Fermion masses arise from Yukawa interactions with the Higgs field after electroweak symmetry breaking \cite{Higgs_BEH_A_1964, Higgs_BEH_B_1964, EnglertBrout_BEH_1964}.

Gauge bosons mediate the fundamental forces. Quantum chromodynamics (QCD) contains eight gluons associated with the symmetry of $SU(3)_{C}$ \cite{Politzer_QCD_1973, GrossWilczek_QCD_1973}. Gluons also carry colour charge and undergo self interaction characteristic of non-Abelian gauge symmetries. In the electroweak sector, the gauge fields associated with $SU(2)_{L}$ and $U(1)_{Y}$ mix after symmetry breaking to produce the physical gauge bosons, the massless photon and the massive weak bosons $W^{\pm}$ and $Z$.

The Higgs field is a scalar $SU(2)_{L}$ doublet whose nonzero vacuum expectation value breaks the electroweak symmetry spontaneously. This mechanism generates masses for the weak gauge bosons and the fermions while preserving gauge invariance. The physical excitation of the field is the Higgs boson, discovered at the LHC in 2012 by the ATLAS and CMS collaborations \cite{ATLAS_Higgs_2012, CMS_Higgs_2012}, which completed the particle spectrum predicted by the SM.

\section{Vector Boson Scattering}
Vector boson scattering (VBS) is an electroweak process in which two incoming quarks emit weak gauge bosons that subsequently interact, as seen in fig. \ref{fig:vbs}. The characteristic topology consists of two forward jets produced by the scattered quarks and two bosons that decay either leptonically or hadronically. The fully hadronic channel, in which both vector bosons decay into quarks, yields the highest expected event rate because the hadronic branching fractions of W and Z bosons are comparatively large, although they differ for WW and ZZ final states and must be treated separately. This channel is experimentally challenging because QCD multijet production generates a dominant background that obscures the diboson signal.


\begin{figure}[h]
    \centering
    \includegraphics[scale=0.4]{figures/vbs.png}
    \caption{Feynman diagram of the vector boson scattering process. Needs to change to better picture}
    \label{fig:vbs}
\end{figure}

An important feature of VBS is its sensitivity to the electroweak symmetry breaking mechanism. In the Standard Model the polarization mode of a gauge boson that points in the direction of its motion originates from the Higgs field, and the scattering amplitude for bosons in this mode shows delicate cancellations between diagrams that contain Higgs exchange and diagrams that contain triple and quartic gauge interactions. These cancellations preserve perturbative unitarity at high energies. Any modification of the electroweak sector may disrupt these cancellations, leading to an enhanced scattering rate or distortions in kinematic distributions.
\newpage
VBS can thus be a probe of the electroweak sector and a channel for detecting possible deviations from the SM. In the fully hadronic decay, information from the boosted large-R jets and forward small-R jets together with jet substructure information can help distinguish VBS signal from background.
\section{Beyond the Standard Model}
Although the SM accurately describes particle interactions at currently accessible energies, it is not a complete theory. It cannot incorporate gravity, for which General Relativity is still our most accurate description \cite{Einstein_GR_1916, vanNieuwenhuizen_Gravity_1981}. It also does not account for dark matter or dark energy \cite{Riess_DE_1998, RubinFordThonnard_DM_1980}, the observed matter–antimatter asymmetry \cite{Gavela_CP_1994}, and must be extended to accommodate nonzero neutrino masses inferred from neutrino oscillations \cite{Fukuda_Nu_1998, Ahmad_SNO_2001, Yanagida_Nu_1980}.

These shortcomings of the SM suggest the existence of new physics at energies beyond the reach of current experiments. Many theoretical extensions predict modified interactions among electroweak gauge bosons or introduce new heavy states that alter scattering amplitudes. Because direct production of such states may be suppressed, precision measurements of processes that are sensitive to the structure of the electroweak sector, such as VBS, provide an indirect but powerful strategy for probing new physics. Deviations in VBS kinematics, polarisation fractions or cross sections may therefore serve as early indicators of physics beyond the SM.
\section{Effective Field Theory}
Effective field theory (EFT) provides a systematic method to describe low energy physics without specifying details of high energy scale degrees of freedom \cite{AppelquistCarazzone_EFT_1975}. If new particles exist at a mass scale larger than the energies probed experimentally, their effects can be encoded in a series of operators. The coefficients of these operators capture how the heavy physics influences observable processes.

A classical example is the Fermi theory of weak interactions \cite{Fermi_Beta_1933}. Long before the discovery of the W boson, beta decay and other weak processes were successfully described by a four-fermion interaction. In modern terms this interaction arises from integrating out the W boson at energies far below its mass. Although the Fermi theory is non renormalizable, it is predictive within its domain of validity, and it enabled precise measurements long before the mediator of the weak force was directly observed. In modern particle physics EFT techniques are used widely to study potential deviations from the Standard Model. 
\subsection{Standard Model Effective Field Theory}
The Standard Model Effective Field Theory (SMEFT) extends the SM by introducing higher-dimensional, gauge-invariant operators constructed from SM fields and respecting the full gauge symmetry \(SU(3)_{C}\times SU(2)_{L}\times U(1)_{Y}\) \cite{BrivioTrott_EFT_2019}.  
With baryon and lepton number conserved, the Lagrangian takes the form
\begin{equation}
\mathcal{L}_{\mathrm{SMEFT}}
=
\mathcal{L}_{\mathrm{SM}}
+ \sum_{i}\frac{c_{i}}{\Lambda^{2}}\,\mathcal{O}^{(6)}_{i}
+ \sum_{j}\frac{d_{j}}{\Lambda^{4}}\,\mathcal{O}^{(8)}_{j}
+ \cdots ,
\end{equation}
\newpage
The suppression of each term is set by \(\Lambda^{4-d}\) for operators of dimension \(d\).  
Dimension-6 operators modify Higgs gauge and triple gauge interactions, but after EWSB, introduce quartic terms, correlated with triple gauge couplings. Independent quartic gauge structures, arise only at dimension-8.

To illustrate how dimension-8 operators enter observables, consider a process with
\(\mathcal{M}=\mathcal{M}_{\mathrm{SM}}+d\,\mathcal{M}_{8}\), where \(d=d_{j}/\Lambda^{4}\) corresponds to a single dimension-8 operator.  
The squared matrix element is
\begin{equation}
|\mathcal{M}|^{2}
=
|\mathcal{M}_{\mathrm{SM}}|^{2}
+ 2d\,\Re\!\left(\mathcal{M}_{\mathrm{SM}}^{*}\mathcal{M}_{8}\right)
+ d^{2}|\mathcal{M}_{8}|^{2},
\end{equation}
leading to a linear interference between the SM and BSM term, and a pure BSM quadratic term.

\subsection{Anomalous Quartic Gauge Couplings}
Quartic gauge interactions are fixed in the SM by electroweak symmetry. Dimension-8 operators introduce new local \(VVVV\) structures that are not constrained by the SM relation between triple and quartic gauge couplings, leading to anomalous quartic gauge couplings (aQGCs). Their amplitudes grow with energy as
\[
\mathcal{M}_{8}\propto \frac{E^{4}}{\Lambda^{4}},
\]
which enhances vector boson scattering at high energies and leads to characteristic changes in angular and kinematic distributions.

\subsection{S, M and T operators}