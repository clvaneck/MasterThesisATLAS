\chapter*{Introduction}
\addcontentsline{toc}{chapter}{Introduction}
\markboth{INTRODUCTION}{Introduction}
The effort to understand the fundamental structure of matter has driven scientific progress since the start of civilization. Ideas about the composition of matter have grown into the quantitative field of particle physics. Modern particle physics combines quantum field theory with large scale accelerator experiments capable of probing interactions beyond those accessible in any other scientific context. The most accurate framework describing all known elementary particles and their interaction is the Standard Model (SM). At the centre of testing this framework is CERN, where high-energy proton beams are brought into collision to study rare and energetic processes.

CERN operates a chain of accelerators that step by step increase the proton energy before injection into the Large Hadron Collider (LHC). The LHC currently delivers proton-proton collisions at a centre of mass energy of $13.6\,\mathrm{TeV}$ at multiple detector collision points. ATLAS and CMS are general purpose detectors designed to study a wide range of physics. LHCb focuses on heavy flavour dynamics, while ALICE studies the quark gluon plasma created in heavy ion collisions. The work presented in this thesis was done as part of the ATLAS collaboration.

This thesis concentrates on vector boson scattering (VBS), a rare electroweak process in which pairs of W or Z bosons are produced in association with two forward jets. The electroweak sector of the Standard Model predicts self interactions between gauge bosons through triple and quartic gauge couplings. Potential modifications of these gauge interactions, for example through anomalous quartic couplings or new heavy states, can alter the scattering kinematics and lead to observable deviations from Standard Model predictions.

VBS signatures can be studied in several decay channels. This thesis examines the fully hadronic mode, which provides significantly larger statistics than leptonic decays. Hadronic V decays can appear as resolved, semi resolved, or unresolved topologies. The present analysis focuses on the unresolved (boosted) regime, where the decay of the two vector bosons shows up as two large-R jets, accompanied by two forward jets separated by a large rapidity gap. This topology is obscured by overwhelming background, namely from Quantum Chromodynamics (QCD) multijet production, making signal extraction non trivial.

To enhance sensitivity, this analysis uses machine learning (ML) methods. Simple cut based selections often fail to capture correlations between jet kinematics, substructure, and overall event topology that distinguishes VBS from background. ML models, in particular deep neural networks and transformer architectures, can process many observables simultaneously and learn complex, multidimensional patterns. The attention mechanism in transformers is especially useful, because it allows the model to identify relationships between all inputs.  
\clearpage
The goal of this thesis is to study event classification in the fully hadronic VBS final state using advanced ML techniques. Chapter~\ref{chap:sm} presents the theoretical framework of the Standard Model together with its extensions and introduces the mechanisms underlying vector boson scattering, quantum chromodynamics, and anomalous quartic gauge couplings. Chapter~\ref{chap:ml} discusses the theory and architecture behind the deep neural networks (DNN), recurrent neural networks (RNN), and transformer models used in this analysis. Chapter~\ref{chap:atlas} shows how the reconstruction of events inside the ATLAS detector is done. Chapter~\ref{chap:montecarlo} discusses how data and event generation was used in the analysis. Chapter~\ref{chap:methods} presents the design and testing of the ML Models used. Chapter~\ref{chap:results} goes over the results. Chapter~\ref{chap:outlook} provides an outlook on the future of this analysis. Chapter~\ref{chap:conclusion} summarizes the work to a conclusion. Chapter~\ref{chap:discussion} gives a discussion on these results.